\chapter{Spiele zur Gruppeneinteilung}
\index{Gruppeneinteilung}

\section{Fäden ziehen}
\index{Fäden ziehen}
\paragraph{Art:} zufällige Pärchenbildung
\paragraph{Ziel:} alle ziehen an einem Ende eines Fadenbüschels
\paragraph{Dauer:} 2--3 Minuten
\paragraph{Wir brauchen dazu:} pro 2 Teilnehmerinnen ei\-nen etwa 1~m langen Bindfaden (zum Beispiel Paketschnur)
\paragraph{So geht es:} Die Moderatorin hält alle Fäden mit einer Hand in der Mitte hoch, so dass ganz viele Fadenenden herunterhängen. Dann greifen sich alle Teilis je ein Fadenende. Die Teilis, (nach dem Entwirren) die zwei Enden jeweils eines Fadens erwischt haben, gehören danach zusammen.
\paragraph{Besondere Hinweise:} Dieses Spiel ist natürlich nur bei einer geraden Anzahl Teilis sinnvoll.
\paragraph{Wann einsetzen:} Um zufällige Pärchen zu bilden.

\section{Süßigkeiten ziehen}
\index{Süßigkeiten ziehen}
\paragraph{Art:} zufällige Gruppenbildung
\paragraph{Ziel:} alle nehmen sich Süßigkeiten und teilen sich so in Gruppen ein
\paragraph{Dauer:} 2--3 Minuten
\paragraph{Wir brauchen dazu:} verschiedenartige Süßwaren (z.\,B.~Weingummi), pro Teili 1 Stück, die Süßigkeiten für die gleiche Gruppe sehen jeweils gleich aus
\paragraph{So geht es:} Die Moderatorin hat vorher die Süßigkeiten entsprechend sortiert. Dann darf sich jede Teilnehmerin ein Teil nehmen. Je nach Art des Teils bestimmt sich die Gruppenzugehörigkeit. Beispiel: Grünes Weingummi ist Gruppe 1, rotes Weingummi Gruppe 2 usw.
\paragraph{Besondere Hinweise:} Die Teilis sollten sich die Süßigkeiten merken, bevor sie sie essen!

Lakritz ist nicht so gut für dieses Spiel geeignet, weil es viele Leute nicht mögen oder dagegen allergisch sind. Gelatine ist problematisch, wenn Veganerinnen mitspielen.
\paragraph{Wann einsetzen:} Um zufällige Gruppen zu bilden.

\section{Standpunkte}
Auf Seite \pageref{standpunkte} zu finden.

